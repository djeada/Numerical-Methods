\documentclass{article}
\usepackage{graphicx}
\usepackage{amsmath} 

\usepackage{pgfplots}
\usepackage{tikz}
\usepackage{nicefrac}
\pgfplotsset{every axis legend/.append style={
at={(0,0)},
anchor=north east}}
\usetikzlibrary{shapes,positioning,intersections,quotes}

\definecolor{darkgreen}{rgb}{0.0, 0.6, 0.0}
\definecolor{darkred}{rgb}{0.7, 0.0, 0.0}

\title{Interpolation}
\begin{document}
\def\horzbar{\text{magic}}

  \pagenumbering{gobble}
  \maketitle
  \newpage
  \pagenumbering{arabic}

\section*{Introduction}

We have to arrays of numbers $X$ and $Y$. Array $X$ contains independent data points. Array $Y$ contains dependent data points $y_i,i=1,…,m$.

We want to find a function $\hat{y}(x)$, which gets the exact same value with given points.\\


\section*{Linear Interpolation}

Linear interpolation is achieved by connecting two data points with a straight line.

For $x_i < x < x_{i+1}$:

$$\hat{y}(x) = y_i + \frac{(y_{i+1} - y_{i})(x - x_{i})}{(x_{i+1} - x_{i})}.$$

\section*{Derivation}

 
\begin{tikzpicture}

	\draw [dashed]  (-3, -1.9) -- (3, -1.9);
	\draw [dashed]  (3, -5) -- (3, -1.9);
	\draw [-stealth](-3,-5) -- (9,-5);
	\draw [-stealth](-3,-5) -- (-3,1);

    \draw [stealth-stealth, blue](0,-3) -- (8,0);
    \draw [stealth-stealth, red](8,0) -- (8,-3);
    \draw [stealth-stealth, green](0,-3) -- (8,-3);
    \draw [stealth-stealth, black](0,-3.1) -- (3,-3.1);

	\node[above right=0pt of {(8, 0)}, outer sep=2pt,fill=none] {$(x_2, y_2)$};
	\node[above right=0pt of {(8, -1.8)}, outer sep=2pt,fill=none, darkred] {$y_2 - y_1$};
	\node[above right=0pt of {(4.5, -3.6)}, outer sep=2pt,fill=none, darkgreen] {$x_2 - x_1$};
	\node[above right=0pt of {(1, -3.6)}, outer sep=2pt,fill=none] {$x - x_1$};
	\node[above right=0pt of {(-1,-3.7)}, outer sep=2pt,fill=none] {$(x_2, y_2)$};
	\node[above right=0pt of {(-1,-1.7)}, outer sep=2pt,fill=none] {$y$};
	\node[above right=0pt of {(3,-4.2)}, outer sep=2pt,fill=none] {$x$};
	\node[above right=0pt of {(3,-2.7)}, outer sep=2pt,fill=none] {$h$};

\end{tikzpicture}


$$\alpha = \frac{y_2 - y_1}{x_2 - x_1}$$

$$h = \alpha \cdot (x - x_1)$$

$$y = y_1 + h$$

$$y = y_1 + (x - x_1) \cdot \frac{y_2 - y_1}{x_2 - x_1}$$

\section*{Example}

We are given two points A(-2, 0) and B (2, 2).\\

\begin{tikzpicture}
    \begin{axis}[
        axis x line=middle,
        axis y line=middle,
        width=8cm,
        height=8cm,
        xmin=-5,   % start the diagram at this x-coordinate
        xmax= 5,   % end   the diagram at this x-coordinate
        ymin=-5,   % start the diagram at this y-coordinate
        ymax= 5,   % end   the diagram at this y-coordinate
        xlabel=$x$,
        ylabel=$y$,
        legend cell align=left,
        legend pos=south east,
        legend style={draw=none},
        tick align=outside,
        enlargelimits=false]
      % plot the function
      \addplot[domain=-5:5, blue, ultra thick,samples=500] {0.5*x + 1};
      \fill[red] (700,700) circle (3pt);
    	  \fill[red] (300, 500) circle (3pt);
    	  \draw [dashed]  (500, 700) -- (680, 700);
    	  \draw [dashed]  (700, 500) -- (700, 680);
    	  \node[above right=0pt of {(255,510)}, outer sep=2pt,fill=none] {A};
	  \node[above right=0pt of {(655,710)}, outer sep=2pt,fill=none] {B};
	\legend{$\nicefrac{1}{2} \cdot x$ + 1}
    \end{axis}
\end{tikzpicture}

Let's try to evaluate the value of the function at $x=1$

$$\hat{y}(x) = y_i + \frac{(y_{i+1} - y_{i})(x - x_{i})}{(x_{i+1} - x_{i})} = 0 + \frac{(2 - 0)(1 - (-2))}{(2 - (-2))} = 1.5$$

\newpage
\section*{Cubic Spline}

The interpolating function in cubic spline interpolation is a set of piecewise cubic functions.\\

For $x_i < x < x_{i+1}$:\\

We have two points $(x_i, y_i)$ and $(x_{i+1}, y_{i+1})$  joined with a cubic polynomial:

$$S_i(x) = a_i x^3 + b_i x^2 + c_i x + d_i$$

For $n$ points, there are $n-1$ cubic functions to find, and each cubic function requires four coefficients ($a_i, b_i, c_i, d_i$).

There are $4(n-1)$ unknowns to find.\\

\begin{tikzpicture}
    \begin{axis}[
        axis x line=middle,
        axis y line=middle,
        width=13cm, height=13cm,     % size of the image
        grid = none,
        grid style={dashed, gray!0},
        %xmode=log,log basis x=10,
        %ymode=log,log basis y=10,
        xmin=-2,     % start the diagram at this x-coordinate
        xmax= 4,    % end   the diagram at this x-coordinate
        ymin=-7,     % start the diagram at this y-coordinate
        ymax= 7,   % end   the diagram at this y-coordinate
        %/pgfplots/xtick={0,1,...,60}, % make steps of length 5
        %extra x ticks={23},
        %extra y ticks={0.507297},
        axis background/.style={fill=white},
        ylabel=y,
        xlabel=x,
        %xticklabels={,,},
        %yticklabels={,,},
        tick align=outside,
        tension=0.08]
      % plot the stirling-formulae
      \addplot[name path global=a, domain=-2:4, blue, thick,samples=500] {-x*x*x + 4*x*x-x-4};
       \fill[red] (211, 39.3) circle (3pt);
       \fill[red] (455, 108.8) circle (3pt);
       \node[above right=0pt of {(211, 39.3)}, outer sep=2pt,fill=none] {$y_1$};
    	  \node[above right=0pt of {(455, 108.8)}, outer sep=2pt,fill=none] {$y_2$};
    	   \node[above right=0pt of {(250, 98.8)}, outer sep=2pt,fill=none] {$S_1(x)$};
\draw [-stealth](280,98) -- (350,82
);
    \end{axis}
\end{tikzpicture}

\section*{Derivation}

We are trying to find a function $S_i(x) = a_i x^3 + b_i x^2 + c_i x + d_i$ going trough both points: $(x_i, y_i)$ and $x_{i+1}, y_{i+1}$.

\begin{align}
S_i(x_i) &= y_i,\quad i = 1,\ldots,n-1,
\end{align}
\begin{align}
S_i(x_{i+1}) &= y_{i+1},\quad i = 1,\ldots,n-1,
\end{align}

Smoothness condition:
\begin{align}
S'_i(x_{i+1}) &= S^{\prime}_{i+1}(x_{i+1}),\quad i = 1,\ldots,n-2,
\end{align}
\begin{align}
S''_i(x_{i+1}) &= S''_{i+1}(x_{i+1}),\quad i = 1,\ldots,n-2,
\end{align}

Boundry condition: The curve is a “straight line” at the end points:
\begin{align}
S''_1(x_1) &= 0
\end{align}
\begin{align}
S''_{n-1}(x_n) &= 0
\end{align}

Let $h_{i}=x_{i}-x_{i-1}$

Let $S_i^{''}(x_i) = S_i^{''}(x_{i+1}) = M_i$

$S_1^{''}(x_1)= M_0 = 0$ and $S_n^{''}(x_n) = M_n = 0$

Other $M_i$ are unknown.

By Lagrange interpolation, we can interpolate each $S''_{i}$ on  $[x_{i-1},x_{i}]$:

$$S''_{i}(x)=M_{i-1}{\frac {x_{i}-x}{h_{i}}}+M_{i}{\frac {x-x_{i-1}}{h_{i}}} \quad for \quad x\in [x_{i-1},x_{i}]$$

Integrating the above equation twice and using the condition that $C_{i}(x_{i-1})=y_{i-1}$ and $ C_{i}(x_{i})=y_{i}$ to determine the constants of integration, we have.

$$ S_{i}(x)=M_{i-1}{\frac {(x_{i}-x)^{3}}{6h_{i}}}+M_{i}{\frac {(x-x_{i-1})^{3}}{6h_{i}}}+\left(y_{i-1}-{\frac {M_{i-1}h_{i}^{2}}{6}}\right){\frac {x_{i}-x}{h_{i}}}+\left(y_{i}-{\frac {M_{i}h_{i}^{2}}{6}}\right){\frac {x-x_{i-1}}{h_{i}}}$$
$${\text{for}}\quad x\in [x_{i-1},x_{i}] $$\\

This expression gives us the cubic spline S(x) if $ M_{i},i=0,1,\cdots ,n$ can be determined.
$$S'_{i+1}(x)=-M_{i}{\frac {(x_{i+1}-x)^{2}}{2h_{i+1}}}+M_{i+1}{\frac {(x-x_{i})^{2}}{2h_{i+1}}}+{\frac {y_{i+1}-y_{i}}{h_{i+1}}}-{\frac {M_{i+1}-M_{i}}{6}}h_{i+1}$$

$$S'_{i+1}(x_{i})=-M_{i}{\frac {h_{i+1}}{2}}+{\frac {y_{i+1}-y_{i}}{h_{i+1}}}-{\frac {M_{i+1}-M_{i}}{6}}h_{i+1}$$

Similarly, when $x\in [x_{i-1},x_{i}]$, we can shift the index to obtain

\begin{align}
S'_{i}(x) &=-M_{i-1}{\frac {(x_{i}-x)^{2}}{2h_{i}}}+M_{i}{\frac {(x-x_{i-1})^{2}}{2h_{i}}}+{\frac {y_{i}-y_{i-1}}{h_{i}}}-{\frac {M_{i}-M_{i-1}}{6}}h_{i}
\end{align}

 
$$ S'_{i}(x_{i})=M_{i}{\frac {h_{i}}{2}}+{\frac {y_{i}-y_{i-1}}{h_{i}}}-{\frac {M_{i}-M_{i-1}}{6}}h_{i}$$

Since $ S'_{i+1}(x_{i})=S'_{i}(x_{i})$, we can derive:

$$\mu _{i}M_{i-1}+2M_{i}+\lambda _{i}M_{i+1}=d_{i}\quad {\text{for}}\quad i=1,2,\cdots ,n-1,$$
 
$$\mu _{i}={\frac {h_{i}}{h_{i}+h_{i+1}}},\quad \lambda _{i}=1-\mu _{i}={\frac {h_{i+1}}{h_{i}+h_{i+1}}},\quad {\text{and}}\quad d_{i}=6f[x_{i-1},x_{i},x_{i+1}]$$

and $f[x_{i-1},x_{i},x_{i+1}]$ is a divided difference.\\

According to different boundary conditions, we can solve the system of equations above to obtain the values of $M_{i}$'s.\\

$S'_{1}(x_{0})=f'_{0}$ and $S'_{n}(x_{n})=f'_{n}$. According to equation (7), we can obtain:

$$S'_{1}(x_{0})=-M_{0}{\frac {(x_{1}-x_{0})^{2}}{2h_{1}}}+M_{1}{\frac {(x_{0}-x_{0})^{2}}{2h_{1}}}+{\frac {y_{1}-y_{0}}{h_{1}}}-{\frac {M_{1}-M_{0}}{6}}h_{1}$$
$$\Rightarrow f'_{0}=-M_{0}{\frac {h_{1}}{2}}+f[x_{0},x_{1}]-{\frac {M_{1}-M_{0}}{6}}h_{1}$$
$$\Rightarrow 2M_{0}+M_{1}={\frac {6}{h_{1}}}(f[x_{0},x_{1}]-f'_{0})=6f[x_{0},x_{0},x_{1}]$$

Analogously:

$$ S'_{n}(x_{n})=-M_{n-1}{\frac {(x_{n}-x_{n})^{2}}{2h_{n}}}+M_{n}{\frac {(x_{n}-x_{n-1})^{2}}{2h_{n}}}+{\frac {y_{n}-y_{n-1}}{h_{n}}}-{\frac {M_{n}-M_{n-1}}{6}}h_{n}$$

$$M_{n-1}+2M_{n}={\frac {6}{h_{n}}}(f'_{n}-f[x_{n-1},x_{n}])=6f[x_{n-1},x_{n},x_{n+1}]$$

Let:\\
$\lambda _{0}=\mu _{n}=1,$\\
$d_{0}=6f[x_{0},x_{0},x_{1}]$ and\\ 
$d_{n}=6f[x_{n-1},x_{n},x_{n}]$

 
\begin{equation*}
  \begin{bmatrix}
    2 & \lambda_0 \\ 
    \mu_1 & 2 & \lambda_1 \\ 
    & \ddots & \ddots & \ddots \\
    && \ddots & \ddots & \ddots \\
	&&& \ddots & \ddots & \ddots \\
	&&&& \mu_{n-1} & 2 & \lambda_{n-1} \\ 
	&&&&& \mu_{n} & 2 \\ 
  \end{bmatrix}
  %
   \begin{bmatrix}
    M_0 \\
    M_1 \\
    \vdots \\
    \vdots \\
    \vdots \\
    M_{n-1} \\
    M_n \\
  \end{bmatrix} 
  =
   %
  \begin{bmatrix}
    d_0 \\
    d_1 \\
    \vdots \\
    \vdots \\
    \vdots \\
    d_{n-1} \\
    d_n \\
  \end{bmatrix} 
\end{equation*}

 \newpage
\section*{Lagrange Polynomial Interpolation}

Lagrange polynomial interpolation gives us a single polynomial that connects all of the data points.

That polynomial is denoted as $L(x)$. It is true that $L(x_i) = y_i$ for all points $(x_i, y_i)$ .

$$L(x) = \sum_{i = 1}^n y_i P_i(x).$$

Each polynomial appearing in the sum is called a Lagrange basis polynomials, $P_i(x)$.

$$P_i(x) = \prod_{j = 1, j\ne i}^n\frac{x - x_j}{x_i - x_j},$$


\section*{Example}

We are given three points A(-1, 1), B(2, 3) and C(3,5).\\

$$P_1(x) = \frac{(x - x_2)(x - x_3)}{(x_1-x_2)(x_1-x_3)} = \frac{(x - 2)(x - 3)}{(-1-2)(-1-3)} = \frac{1}{12}(x^2 - 5x + 6)$$

$$P_2(x) = \frac{(x - x_1)(x - x_3)}{(x_2-x_1)(x_2-x_3)} = \frac{(x + 1)(x - 3)}{(2 + 1)(2-3)} = -\frac{1}{3}(x^2 - 2x - 3)$$

$$P_3(x) = \frac{(x - x_1)(x - x_2)}{(x_3-x_1)(x_3-x_2)} = \frac{(x + 1)(x - 2)}{(3 + 1)(3-2)} =\frac{1}{4}(x^2 -x - 2)$$

$$ L(x) = 1 \cdot P_1(x) + 3 \cdot P_2(x) + 5 \cdot P_3(x) $$

$$ L(x) = 1 \cdot P_1(x) + 3 \cdot P_2(x) + 5 \cdot P_3(x) $$

$$ L(x) = \frac{1}{3} x^2 + \frac{1}{3} x + 1 $$

\begin{tikzpicture}
    \begin{axis}[
        axis x line=middle,
        axis y line=middle,
        width=10cm,
        height=10cm,
        xmin=-5,   % start the diagram at this x-coordinate
        xmax= 6,   % end   the diagram at this x-coordinate
        ymin= -1,   % start the diagram at this y-coordinate
        ymax= 8,   % end   the diagram at this y-coordinate
        xlabel=$x$,
        ylabel=$y$,
        legend cell align=left,
        legend pos=north east,
        legend style={draw=none},
        tick align=outside,
        enlargelimits=false,
        xtick distance=1,
        ytick distance=1]
         
      % plot the function
      \addplot[domain=-5:10, blue, ultra thick,samples=500] {1/3*(x^2 + x + 3)};

      \fill[red] (400, 20) circle (3pt);
    	  \fill[red] (700, 40) circle (3pt);
    	  \fill[red] (800, 60) circle (3pt);

    	  \node[above right=0pt of {(340, 13)}, outer sep=2pt,fill=none] {A};
	  \node[above right=0pt of {(630, 40)}, outer sep=2pt,fill=none] {B};
	 \node[above right=0pt of {(730, 60)}, outer sep=2pt,fill=none] {C};

	\legend{$\frac{1}{3} x^2 + \frac{1}{3} x + 1$}

    \end{axis}
\end{tikzpicture}

\end{document}