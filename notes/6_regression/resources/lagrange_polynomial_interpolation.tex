\documentclass{article}
\usepackage{graphicx}

\usepackage{pgfplots}
\usepackage{tikz}
\pgfplotsset{every axis legend/.append style={
at={(0,0)},
anchor=north east}}
\usetikzlibrary{shapes,positioning,intersections,quotes}

\definecolor{darkgreen}{rgb}{0.0, 0.6, 0.0}
\definecolor{darkred}{rgb}{0.7, 0.0, 0.0}

\title{Simpson's Method}
\begin{document}

\begin{tikzpicture}
    \begin{axis}[
        axis x line=middle,
        axis y line=middle,
        width=10cm,
        height=10cm,
        xmin=-5,   % start the diagram at this x-coordinate
        xmax= 6,   % end   the diagram at this x-coordinate
        ymin= -1,   % start the diagram at this y-coordinate
        ymax= 8,   % end   the diagram at this y-coordinate
        xlabel=$x$,
        ylabel=$y$,
        legend cell align=left,
        legend pos=north east,
        legend style={draw=none},
        tick align=outside,
        enlargelimits=false,
        xtick distance=1,
        ytick distance=1]
         
      % plot the function
      \addplot[domain=-5:10, blue, ultra thick,samples=500] {1/3*(x^2 + x + 3)};

      \fill[red] (400, 20) circle (3pt);
    	  \fill[red] (700, 40) circle (3pt);
    	  \fill[red] (800, 60) circle (3pt);

    	  \node[above right=0pt of {(340, 13)}, outer sep=2pt,fill=none] {A};
	  \node[above right=0pt of {(630, 40)}, outer sep=2pt,fill=none] {B};
	 \node[above right=0pt of {(730, 60)}, outer sep=2pt,fill=none] {C};

	\legend{$\frac{1}{3} x^2 + \frac{1}{3} x + 1$}

    \end{axis}
\end{tikzpicture}

\end{document}
