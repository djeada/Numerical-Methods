\documentclass{article}
\usepackage{graphicx}
\usepackage{nicefrac}
\usepackage{pgfplots}
\usepackage{tikz}
\pgfplotsset{every axis legend/.append style={
at={(0,0)},
anchor=north east}}
\usetikzlibrary{shapes,positioning,intersections,quotes}

\definecolor{darkgreen}{rgb}{0.0, 0.6, 0.0}
\definecolor{darkred}{rgb}{0.7, 0.0, 0.0}

\title{Simpson's Method}
\begin{document}

\begin{tikzpicture}
    \begin{axis}[
        axis x line=middle,
        axis y line=middle,
        width=8cm,
        height=8cm,
        xmin=-5,   % start the diagram at this x-coordinate
        xmax= 5,   % end   the diagram at this x-coordinate
        ymin=-5,   % start the diagram at this y-coordinate
        ymax= 5,   % end   the diagram at this y-coordinate
        xlabel=$x$,
        ylabel=$y$,
        legend cell align=left,
        legend pos=south east,
        legend style={draw=none},
        tick align=outside,
        enlargelimits=false]
      % plot the function
      \addplot[domain=-5:5, blue, ultra thick,samples=500] {0.5*x + 1};
      \fill[red] (700,700) circle (3pt);
    	  \fill[red] (300, 500) circle (3pt);
    	  \draw [dashed]  (500, 700) -- (680, 700);
    	  \draw [dashed]  (700, 500) -- (700, 680);
    	  \node[above right=0pt of {(255,510)}, outer sep=2pt,fill=none] {A};
	  \node[above right=0pt of {(655,710)}, outer sep=2pt,fill=none] {B};
	\legend{$\nicefrac{1}{2} \cdot x$ + 1}
    \end{axis}
\end{tikzpicture}


\end{document}
